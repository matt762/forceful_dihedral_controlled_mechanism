\documentclass[lettersize,journal]{IEEEtran}
\usepackage{amsmath,amsfonts}
\usepackage{algorithmic}
\usepackage{algorithm}
\usepackage{array}
\usepackage[caption=false,font=normalsize,labelfont=sf,textfont=sf]{subfig}
\usepackage{textcomp}
\usepackage{stfloats}
\usepackage{url}
\usepackage{verbatim}
\usepackage{graphicx}
\usepackage{cite}
\hyphenation{op-tical net-works semi-conduc-tor IEEE-Xplore}
% updated with editorial comments 8/9/2021

\begin{document}

\title{Forceful dihedral-controlled mechanism for bird inspired drones}

\author{Mattéo Fiore, Hoang-Vu Phan, Dario Floreano,~\IEEEmembership{Fellow,~IEEE,}
        % <-this % stops a space
\thanks{The authors are with
the Laboratory of Intelligent Systems, Ecole Polytechnique Federale de Lausanne (EPFL), CH1015 Lausanne, Switzerland.}}% <-this % stops a space

% The paper headers
% Remember, if you use this you must call \IEEEpubidadjcol in the second
% column for its text to clear the IEEEpubid mark.

\maketitle

\begin{abstract}
Describe your work briefly: what is the problem, what others did, what you contributed, experimental results/comparisons, and conclusion.
\end{abstract}

\begin{IEEEkeywords}
keyword 1, keyword 2, keyword 3, keyword 4.
\end{IEEEkeywords}

\section{Introduction}
\IEEEPARstart{T}{}his section should include the following information clearly and concisely:
\begin{enumerate}
    \item 1-2 sentences: clearly state the problem
    \item State of the art: what other researchers did to alleviate the problem and why those contributions do not fully solve the problem, related solutions and their shortcomings with citations (e.g. \cite{Bircheretal}, \cite{Oettershagenetal})
    \item 1-2 sentences stating the proposed solution (Here we proposed a new method for… / a new mechanical design for… / a new hypothesis for explaining… and show that [explain main result])
    \item Add one figure showing the final solution (if applicable)
\end{enumerate}

\IEEEPARstart{B}{}irds possess very large pectoralis muscles (8-11 \% body mass) to primarily drive wing movements around the shoulder joints for
powering flapping flight as well as controlling wing dihedral during gliding maneuvers. Inspired by birds, this project aims to
develop high-torque constrained mechanism that can forcefully and swiftly change the wing dihedral and can resist high aerodynamic
wing load during flight.

\section{Method}
The following should be mentioned here:
\begin{enumerate}
    \item 1-2 sentences: describe concisely essential aspects of method and/or algorithm and/or mechanical design and/or new hypothesis
    \item Articulate detailed explanation of method in subsections if needed
    \item Insert figures with captions that comprehensively explain what the figure shows and explain variables (if any). A reader should be able to understand the contents of the figure without reading the main text of the report
    \item In main text, do not write “Figure X shows this and that”; instead explain the method/design/interface/etc as if there was no figure and cite the figure in brackets at the end of the statements. A reader should be able to understand the paper without necessarily looking at the figures
\end{enumerate}

\section{Results}
Here you would do the following:
\begin{enumerate}
    \item 1-2 sentences: describe what you want to measure and how you perform the measures, also known as experimental method (not to be confused with the Method described in the previous section)
    \item Describe the results in a concise manner and provide numbers that support the results. Articulate the results section in subsections if necessary.
    \item If you report averages, give standard error in brackets; if you state “condition x is superior to condition y” give numbers in brackets and possibly a statistical test
    \item In main text, do not write “Figure X shows this and that”; instead explain the concept or results as if there was no figure and cite the figure in brackets at the end of the statements.
\end{enumerate}

\section{Conclusion}
Here you do the following:
\begin{enumerate}
    \item 1-2 sentences summarising the main highlight of your novel solution and results
    \item Discuss weaknesses and remaining work
    \item 1 sentence to describe the impact of your work (imagine to answer the “so what?” question of a reader); e.g., “the design method for self-folding pasta described here could make pasta packaging much more compact and reduce transportation costs”.
\end{enumerate}
\bibliographystyle{IEEEtran}
\bibliography{IEEEabrv,IEEEexample}

\section*{Appendix}
Describe concisely any additional material that you have put on public repositories (e.g., software on Github, data on Zenodo, videos on YouTube or your own website unless there is a confidentiality agreement)

The following are additional writing tips:
\begin{itemize}
    \item Decide on all titles and subtitles
    \item Put figures (or placeholder figures)
    \item Write the caption (the main idea of your work should be understandable by just looking at the figures)
    \item Write content in the form of bullet points in each subsection
    \item Turn the bullet points into full text
    \item Write the abstract last
    \item When turning bullet points into text, write it like you would explain it to a friend. And then iteratively make it more scientific. This is usually easier than just writing a full sentence from scratch.
    \item Use Grammarly
    \item The text should be understandable independently of the figures
    \item Remove vague or subjective words such as
    \begin{itemize}
        \item efficient
        \item clearly shows
        \item is robust
        \item roughly
    \end{itemize}
    \item Don't use "we" and write in an impersonal form.
\end{itemize}

\end{document}


